\begin{task}
    \textbf{1.} Пусть $F: \mathbb{R}^2 \rightarrow \mathbb{R}$ задано равенством $F\left(x_1, x_2\right)=\sqrt[3]{x_1^2 x_2}$. 
    Показать, что $F$ имеет вариацию по Лагранжу, но не дифференцируемо по Гато в нуле. 
    \textbf{2.} Пусть $X$ --- бесконечномерное нормированное пространство, $F: X \rightarrow \mathbb{R}$ 
    --- линейный неограниченный функционал. Показать, что $F$ имеет вариацию по Лагранжу в нуле, 
    но не дифференцируемо по Гато.
    
    \textbf{Решение.} \textbf{1} Пусть $h = \left( h_1, h_2 \right)$. 
    Тогда $F\left(t h_1, t h_2\right)=\sqrt[3]{\left(t h_1\right)^{2} t h_2}=t \sqrt[3]{h_1^2 h_2}$. Значит,
    \[ 
    \frac{F\left(t h_1, t h_2\right)-F(0,0)}{t}=\sqrt[3]{h_1^2 h_2}, \quad F^{\prime}(0,0)\left[\left(h_1, h_2\right)\right]=\sqrt[3]{h_1^2 h_2}
    \]  
    легко видеть, что это отображение нелинейно.\\
    \textbf{2} Если функционал $F$ линеен, то $F(t h)-F(0)=t F(h)$; значит, $F^{\prime}(0)[h]=F(h)$. Это отображение линейно, но разрывно.
    
    \end{task}