\begin{task}
    Рассмотрим задачу
    \begin{equation*}
        \int_{-T_0}^{T_0} x \sqrt{\dot{x}^2+1} d t \rightarrow \min , 
        x\left(T_0\right) = x \left( -T_0 \right) = \xi.
    \end{equation*}
    \begin{enumerate}
        \item Выписать уравнение Якоби, подобрать одно из его решений, 
        затем найти общее решение. 
        \item Пусть допустимых экстремалей две. Доказать, что одна из них 
        является точкой сильного минимума, а вторая не является точкой слабого минимума.
    \end{enumerate}


    \textbf{Peшение.} 
    \begin{definition}
        Скажем, что выполнено усиленное условие Лежандра, 
        если $\widehat L_{\dot{x}\dot{x}} > 0 \; \forall t \in [-T_0, T_0]$.
    \end{definition}

    \begin{definition} Скажем, что выполнено условие Якоби, 
        если справедливо усиленное условие Лежандра, а решение уравнения Якоби
        \begin{equation} \label{eqJacobi28} 
            -\frac{d}{d t}\left(\widehat{L}_{\dot{x} \dot{x}}(t) \dot{h}
                +\widehat{L}_{\dot{x} x}(t) h\right)
                +\left(\widehat{L}_{\dot{x} x}(t) \dot{h}
                +\widehat{f}_{x x}(t) h\right)=0 \quad 
                \Leftrightarrow \quad \frac{d}{d t}\left(\widehat{L}_{\dot{x} \dot{x}}(t) \dot{h}\right)
                =\left(\widehat{L}_{x x}(t)-\frac{d}{d t} \widehat{L}_{\dot{x} x}(t)\right) h
        \end{equation}
        не обращается в ноль на интервале $\left(t_0, t_1\right)$ при начальных условиях: 
        $h\left(t_0\right)=0, \quad \dot{h}\left(t_0\right)=1$. 
    \end{definition}
    \begin{definition}
        Скажем, что выполнено усиленное условие Якоби, если справедливо усиленное условие 
        Лежандра, а решение уравнения \eqref{eqJacobi28} не обращается в ноль на полусегменте
        $(-T_0, T_0]$ при начальных условиях: ${h(-T_0)=0, \; \dot{h}(-T_0)=1}$.
    \end{definition}

    \begin{definition}
        Скажем, что выполнено усиленное условие Вейерштрасса, 
        если функиия $\dot{x} \mapsto L(t, x(t), \dot{x})$ 
        выпукла в '$C$'-окрестности экстремали $\widehat{x}$ 
        при любом $t \in\left[-T_0, T_0\right]$, т.е. для любого 
        $t \in\left[-T_0, T_0\right]$ u $x(t) \in 
        \mathcal{O}(\widehat{x}, \varepsilon) \;
        (\text {с некоторым } {\varepsilon>0})$ функиия
        $\dot{x} \mapsto L(t, x(t), \dot{x})$ выпукла.
    \end{definition}

    \begin{theorem}
        Если выполнены усиленное 
        условие Якоби и усиленное условие Вейерштрасса, то экстремаль доставляет
        сильный максимум.
    \end{theorem}




1. Уравнение Якоби в данном случае имеет вид:
\begin{equation*}
    \ddot{h}-\frac{2}{C} \th (\frac{t}{C}) \dot{h}+\frac{1}{C^2} h = 0    
\end{equation*}
Оно имеет два линейно независимых решения: 
$h_1(t)=\operatorname{sh} \frac{t}{C}$ и $h_2(t)=\operatorname{ch} \frac{t}{C}-\frac{t}{C} \operatorname{sh} \frac{t}{C}$. 
Общее решение, подчиненное условию $h(-1)=0$, таково:

\begin{equation*}
h(t)=\left(\operatorname{ch} \frac{t}{C}-\frac{t}{C} \operatorname{sh} \frac{t}{C}\right) \operatorname{sh} \frac{1}{C}+\left(\operatorname{ch} \frac{1}{C}-\frac{1}{C} \operatorname{sh} \frac{1}{C}\right) \operatorname{sh} \frac{t}{C} .
\end{equation*}

2. Было показано (см. \ref{task6}), что экстремали существуют, 
если и только если $\xi \geq \xi_{*}=\operatorname{sh} \frac{1}{C_0}$, 
где $C_0=\operatorname{th} \frac{1}{C_0} \approx 1.5088 \ldots$ При этом, экстремаль задается формулой 
$x(t)=C \operatorname{ch} \frac{t}{C}$, а параметр $C>0$ есть корень уравнения $\varphi(C) = \xi$, 
где $\varphi(C) \stackrel{\text { def }}{=} C \operatorname{ch} \frac{1}{C}$. 

Функция $C \mapsto \varphi(C)$ выпукла, т.к. $\varphi^{\prime \prime}(C)=C^{-3} \operatorname{ch} \frac{1}{C}$. 
Ее минимум достигается в точке $C_0$, где $\varphi^{\prime}\left(C_0\right)=0$. Отметим, что
$$
\varphi^{\prime}(C)=\left[\operatorname{ch} \frac{1}{C}-\frac{1}{C} \operatorname{sh} \frac{1}{C}\right]
$$

Если $\xi > \xi_{*}$, то существуют ровно два значения $C_1 \in\left(0, C_0\right)$ и $C_2>C_0$, 
которые удовлетворяют условию $\varphi(C)=R$. Покажем, что экстремаль $\widehat{x}_2=C_2 \operatorname{ch} \frac{1}{C_2}$ 
доставляет сильный (локальный) минимум, а экстремаль $\widehat{x}_1=C_1 \operatorname{ch} \frac{1}{C_1}$ 
не является ни слабым минимумом, ни слабым максимумом. Прежде всего, отметим, что для обеих экстремалей выполнено 
усиленное условие Лежандра, а именно: $\widehat{f}_{\dot{x} \dot{x}}(t)=C \mathrm{ch}^{-2} \frac{t}{C} > 0$ 
и потому ни одна из них не является локальным максимумом. \par


Так как $h(0)=\operatorname{sh} \frac{1}{C} \neq 0$, то нули функции $h$ совпадают с нулями функции
$$
z: t \mapsto z(t)=\frac{h(t)}{\operatorname{sh} \frac{1}{C} \operatorname{sh} \frac{t}{C}}=\left(\operatorname{cth} \frac{t}{C}-\frac{t}{C}\right)+\left(\operatorname{cth} \frac{1}{C}-\frac{1}{C}\right)
$$

Заметим, что $z^{\prime}(t)<0$, а $z(1) = \frac{2 \varphi^{\prime}(C)}{\operatorname{sh} \frac{1}{C}}$. 
Поэтому, если $z(1)<0 \Leftrightarrow C=C_1$, то условие Якоби не выполнено, а если $z(1) > 0 \Leftrightarrow C=C_2$, 
то выполнено усиленное условие Якоби.

\end{task}